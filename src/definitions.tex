\startcomponent definitions

\chapter{Definitions}

The objective of the well-told story is to establish and maintain \emph{
	verisimilitude} in the mind of its reader. An admittedly pretentious word, but
one that is born from a desperate need to describe the ineffable experience
that sits on the boundary of mind and body; that quality a reader obtains by
\quotation{suspending their disbelief} and allowing the substance of a story to
temporarily subsume the substance of reality.

The concept of verisimilitude can only ever be dubiously defined, so please
forgive my failure to do so here. Instead, the conceit of my words is to play a
certain language game with you. A game that is played by analysis, example, and
advice. One that seeks to encode the contents of imagination by using the
limited language that can be written on its boundary. By deconstructing, then
reconstructing, then deconstructing again the various elements that give
structure to our stories, we can hope to elucidate the true definition of
verisimilitude by powerful, albeit long-winded vibes.

At the risk of losing both your attention and your respect, I am obliged
therefore to address two subjects immediately: what this text is promising you,
and why I am speaking to you directly.

First, my intention for this text is selfish. The nature of verisimilitude is
tantalizingly difficult to capture, making it infinitely compelling to
contemplate. Yet the subject is a black hole where questions consistently
swallow my thoughts whole. Thus, these words act to tether my contemplation to
something at least resembling progress. The theory is that this outcome has a
non-zero probability of being useful to others. A weak motivation for you, to
be sure, but I'm mostly obfuscating for the sake of dramatic effect. I don't
want to spoil anything.

My second obligation is why you are a character in this discussion. You
probably don't appreciate it. By pupetting the word \quote{you} I, in a sense,
am putting words in your mouth. But this perspective has utility in discussing
what I consider a fundamental definition in the art of language: that of
semantics.

% TODO: reference Wittgenstein.

Language is a social construct built from context-driven games. The equilibria
of these games constitute narratives, and relative to a given context, these
narratives are embued with meaning. This meaning has structure that we will
call \emph{semantics}. Observe that as you squint at these phrases to discern
their meaning, we are engaged in this very game. You ask, \quotation{what am I
	reading,} followed by \quotation{what does the author mean by this,} followed
by \quotation{what does the author mean by drawing attention to what I should
	be thinking?} Immediately we are struck by a fascinating insight: we see that
semantics are structured in levels of strategy. As I write, the meaning of the
text, from my perspective, is encapsulated in an infinite regression of
inquiry:

\startitemize
\item First-order: \quotation{what do I think it means?}

\item Second-order: \quotation{what do I think you think it means?}

\item Third-order: \quotation{what do I think you think I think it means?}

\item Fourth-order: \quotation{what do I think ...?}
\stopitemize

\noindent Later, I shall quiz you on what you think is meant by sixteenth-order
semantics.

From your point of view, the structure of semantics is identical, but where I
must care about what I think you think I mean by my words, you generally only
have to worry about what you think I mean. In other words, I must consider up
to third-order semantics in deciding what to write, but a reader generally only
needs to consider first and second-order semantics in deciding how to read. I
say \quote{generally} because most written narratives don't involve direct
address. Right now, you are burdened by an additional layer of semantics and
for that I apologize. If it makes you feel better, I too am burdened by an
additional layer of semantics. Namely, I have to conjecture on how I think you
think I think you think, and I assure you that it is taxing.

This relationship between author and reader can be made more explicit by
observing that each order of semantics, as analogized above, begin with the
same \quotation{what do I think ...} In the transition from sender to receiver,
you can imagine that we are factoring out this initial phrase, decrementing the
order, and tossing the extraneous null-order. This leaves the reader with their
respective regression:

\startitemize
\item Zeroeth-order: \quotation{\overstrike{what does it mean?}}

\item First-order: \quotation{what do you think it means?}

\item Second-order: \quotation{what do you think I think it means?}

\item Third-order: \quotation{what do you think I think you think it means?}

\item Fourth-order: \quotation{what do you think ...?}
\stopitemize

Tragically, the zeroeth-order semantics of a phrase is hidden information.
Recall that language is a social construct, thus words do not have divine
interpretations.

Earlier I used a pecululiar word to describe the orders of semantics: strategy.
If language is a game, you certainly don't wish to lose, thus you employ
strategy to protect yourself from abuse. You do this all the time, even if you
don't know to call it strategy. For example, if you meet a strange man trying
to \quotation{gift} you a book called \emph{The Spirit and Teachings of the
	Enlightened One} then proceeds to hassle you for \quotation{donations,} you
might infer that he wishes to indoctinate you into a cult and make a sale while
he's at it.

A more sinister example would be someone sending you a meme about
\quotation{welfare queens.} You might infer that they are trying to persuade
you into mistaking welfare for theft under the guise of humor. This inferencing
is strategy and you excercise it by evaluating second and third-order
semantics.

In fact, you might even be questioning whether my example just now was an
attempt to persuade you to support social welfare programs. It's true. You're
quite good at this!

\rule

Hopefully, by now, the point has been made and the utility of direct address
has begun to wane---freeing us from this complexity and allowing us to return
to the simple subject of verisimilitude and storytelling. While it's true that
there is an inevitible politics to what we read and write due to the context of
living in a society, the strategies involved in our semantics can be just as
beautiful as it can be sinister. In an intimate context, the second-order
semantics of a message may be \quotation{I love you \emph{for} your flaws, not
	for your lack of them,} just as well as it may be \quotation{I \emph{really}
	think you would benefit from meeting the Enlightened One.}

This is a key observation that will help reveal the elusive mechanics of
verisimilitude. Ultimately, our comprehension of semantics is enabled by our
capacity for empathy---a breathtaking ability of the human mind---and our
allowance for a certain vulnerability. By relaxing our guard, and accepting the
narrative being presented, one can engage their empathy to connect with the
text on its own terms---becoming \quotation{transported} into the world of the
author. The consequences of this decision are profound, powerful, beautiful,
and dangerous.

The intention of this text is to teach you how to acheive this for yourself.

\stopcomponent
