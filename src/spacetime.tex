\startcomponent spacetime

\chapter{Physical Geometry}

Classically, the invocation of a frame of reference, and thus with it the
conception of space-time, is derived through the epiphany of Descartes: \emph{I
	think, therefore I am}. It's clear now that Descartes' beautiful rhetoric can
be constructed formally by reflecting on the axiom of choice: \emph{the
	existence of language is proven by the medium}. But if we are going to be
abandoning the principle of reference frames, how then could it be possible to
construct definitions of space and time that are consistent with physical
geometry?

The human struggle with this question lies in our failure to generalize our
interpretation of translations -- or, stated more plainly, the general problem
of interpretation: our obsessive strive toward the perfection of our language
that is often blocked by the mistaken belief that the language spoken by the
mind to itself is indistinguishable from the language spoken by the voice to
its peers. But by propositional extensionally, if the languages were to be
indistinguishable, then they must be equivalent; however, since these languages
are constructed from different mediums, this is a contradiction. This argument
is called the \emph{mis\`ere condition}, and here it is argued without shame
that this is the only contradiction permitted by God.

Therefore, left with the burden of knowledge, we must invoke choice to define
potential quotient constructions.

Suppose we are contemplating an experiment to calibrate our lab equipment and
endeavor to construct the simplest possible measurement to ensure the
independence of contingent language. We envision then the existence of two
indivisible particles and we define \emph{measurement} to be the construction
of a proof that there exists a language that is exchanged between them.

By choice, we must therefore study the medium using strategic language.

Since our particles are indivisible, we cannot yet use language that is
contingent on the internal structure of the particles themselves. So instead,
we must resort to \emph{nonempty numbers}: natural numbers that are not zero.
If there exists a language between the two particles, then we can identify this
language by observing the medium and associating unique transformations with
unique numbers.

The nonempty numbers form a strategic language equipped with the property of
\emph{universal translation}: the unique ability to translate all alien
languages into formal languages. However, all universal translations are
indistinguishable, thus, by the mis\`ere condition, any translation of an alien
language that is complete is also alien.

We must now define what constitutes the transformation of a medium, and to this
end, we endeavor to construct a formal interpretation of Einstein's special
relativity. Our objective thus turns to the question of defining what is
understood classically as \quotation{simultaneous events,} or contemporaneously
as simply \emph{data}.

By having chosen nonempty numbers for our strategic language, the definition is
now immediate. The catch, however, is that we must now engage the problem of
teleological deduction, or time. The rhetoric of verisimilitude compels us to
avoid using time as a given since we have yet to reach our destination of a
general theory of verisimulation.

As of now, we can only proceed by establishing a \emph{formal analogy}, wherein
a speculative proposal is presented and authenticity is informed by the
difficulty of proving it wrong. Here, it is proposed that all indivisible
particles communicate by transforming a shared medium into a game of coinage.
Clarification will be written in vernacular:

First, one of the two particles bids a nonempty number by translating its
interpretation of the number into the medium. We do not assume anything about
the complexity of the interpretation, nor even the existence of the medium,
thus the numbers being used can only be compared using propositional
extensionality within the context of calculus.

Second, the particle receiving the bid replies with a bid that is distinct from
the original.

Third, and so on, the receiving particle must reply with a nonempty number that
is linearly independent from all numbers bid previously. In other words, the
number that a particle chooses must not be derived from any of the numbers
already declared before it. If one of the particles produces a linearly
dependent bid, we assume that communication is not possible between the
particles -- or, in other words, the particles are assumed to be alien to each
other and choice must be invoked in order to continue analysis.

Assuming this is not the case, the game is otherwise concluded when one of the
particles no longer has any remaining bids. This particle is then
\quotation{the winner who receives the pot,} wherein the final message,
constructed as a unit terminated list of linearly independent nonempty numbers,
is given an orientation that allows the scientist to distinguish the two
particles from one another: one is the winner, the other is the loser --
positive, negative; charming, strange; black, white; us, them; beautiful and
elegant.

In classical language, we have just argued that photons do not travel, they
simply arrive at their destination instantaneously. And if we were to
articulate this further, our reason would eventually rest on the interpretation
that photons are somehow \quotation{equivalent} to some \quotation{aether},
prompting us to once again consider the axiom of choice.

But all translations of alien languages that are complete are also alien, and
thus, the scientist must be considerate in their surveillance of nature and how
they communicate their data to others. As for their pupils, they must learn to
understand how the choices we make to influence ourselves influence the more
general verisimulation of humanity.

\stopcomponent
