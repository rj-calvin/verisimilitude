\startcomponent axioms

\chapter{Axioms and Dogma}

The special theory of verisimilitude is built using the calculus of
constructions as defined by the three core axioms and their natural dogma:

\startitemize
\item Propositional extensionality: \quotation{equivalent languages are
	indistinguishable.}

\item Quotient construction: \quotation{all languages are contingent
	languages.}

\item Choice: \quotation{the existence of language is proven by the medium.}
\stopitemize

By choice, we study the nature of language using \emph{strategic languages}:
languages that are interpreted as translations of other languages. The
definition of \emph{strategic choices} then derives elegantly.

However, since strategic languages are contingent languages by their
construction, their translation -- or, more generally, their
\emph{interpretation} -- may not be unique or be known to exist at all.
Languages that are without interpretation are called \emph{alien languages}.
Languages that have a unique interpretation are called \emph{formal languages}.

With axioms now either accepted or rejected, and with dogma that is
sufficiently persuasive, the special theory of verisimilitude can now be
defined as a general theory of physical constructions.

\stopcomponent
