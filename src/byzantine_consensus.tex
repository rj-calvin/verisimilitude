\startcomponent byzantine_consensus

\chapter{Consensus}

In the study of communication, the Byzantine generals problem is foundational.
The impossibility theorem that it inspires brings madness to all who understand
its scope. However, to the optimist, impossibility is the greatest of news
since knowledge of what is impossible improves the efficiency of our reason.

Before discourse on the general Byzantine problem can begin, it is first
necessary to discuss the simpler problem of only two generals. The more general
problem of the Byzantine generals will be discussed later. Nevertheless, the
simpler analogy is stated as such:

Two generals and their armies find themselves on opposite sides of enemy
territory. The enemy is assumed to be strong enough to repel an attack from one
army, but not a simultaneous attack from both. Any allied travel through enemy
territory is assumed to be treacherous. The two generals are assumed to
recognize the enemy and distinguish them from their allies. Both generals are
aware of the other's initial position, but knows nothing of the other's
movements or behavior. Lastly, it is assumed that the only form of
communication between the allied leaders is done through messengers sent
through enemy territory, and thus, the problem of the two generals asks whether
it is possible for these generals to coordinate a specific time to attack the
enemy in unison.

The key to understanding the problem is to sympathize with one of the generals.
This general, eager to initiate an attack, may choose a time then send a
messenger to inform their peer. But since travel through enemy territory is
treacherous, this initial general cannot be certain that their message
successfully reaches its destination. Without this confidence that their
message has been received, the general may not wish to attack if they fear that
they will do so alone. Likewise, if the message is received by the other
general, they may wish to send a confirmation to resolve this issue of
confidence. But just as the first, the second general cannot be sure that their
confirmation will reach its destination either. And even if this confirmation
does go through, the receiving general might feel the need to confirm the
confirmation to assuage this uncertainty in the other---revealing the dilemma.

In the language of verisimilitude, the problem of the two generals is reasoned
about using \emph{teleological deduction}: a form of deduction that is not
\emph{deterministic}---or, stated more concretely, a strategic language that
uses symbols prior to their definition.

Suppose we reason about the problem backwards by assuming that a successful
attack has already occurred. We are thus required to deduce how such an event
could be possible. We know from deterministic deduction that there is no choice
of protocol that can guarantee a simultaneous attack, therefore, we can surmise
that either the generals accepted some amount of uncertainty in their protocol,
or the attack itself was not simultaneous. Thus, it is revealed that there are
two dimensions to the question posed by the analogy: one that seeks to minimize
uncertainty and one that seeks to maximize simultaneity.

It may seem counter-intuitive to discuss the possibility of the attack not
being simultaneous since it was the mutual fear of idleness between the two
generals that gave rise to the paradox in the first place. Yet, a scientist
might recognize this technique as simply the rejection of a hypothesis,
prompting us to wonder how much we really understand about the problem to begin
with.

The purpose of the analogy of the two generals is to discuss the difficulty of
creating deterministic networking protocols over an unreliable medium, or
\emph{channel}. However, the goal of verisimilitude is to contemplate the more
general communication between particles. In a lab environment, the problem of
the two generals cannot be seen, only the \quotation{results of an attack} can
be seen: either the attack is led to failure by one general, or the attack is
led to success by two. The integrity of the enemy represents the
measurement---or, in the language of verisimilitude: the destruction of the
medium, called \emph{termination,} serves as the proof that a language exists
between the two generals.

From this lens, we see how it might be reasonable to contemplate events that
are not authentically \quotation{simultaneous} in the classical sense.
Extending the analogy, suppose our generals are clever and have sent scouts to
study their enemy. The scouts might then discover that the enemy is equipped
with a large horn to rally troops to defense. If such a discovery is made, a
general informed of this alarm could reason that they no longer need to
communicate with their peer in order to coordinate an attack. Instead, if the
general simply initiates the attack immediately, the sound of the enemy's horn
will signal their ally to join the offensive.

However, this line of reasoning should be insulting to the educated reader.
Extending an analogy is not a rigorous exercise. But the subsequent
interpretation, however, can be made rigorous if it is eventually written in a
formal language.

The practice of teleological deduction, then, seems to demand of us a certain
humility. An admission to the incompleteness of any language invented by the
mind that is destined for the interpretation of others. Much like the two
generals, our languages that are deterministic are also symbolic. Thus, the
generals have no choice but to send messengers if they desire certainty in
their coordination. In the language of verisimilitude, this certainty is called
\emph{consensus.}

\stopcomponent
